\title{CS 435 - Computational Photography}
\author{
        Assignment 2 - High Dynamic Range (HDR) Images
}
\date{}
\documentclass[12pt]{article}
\usepackage[margin=0.7in]{geometry}
\usepackage{graphicx}
\usepackage{float}
\usepackage{amsmath}


\begin{document}
\maketitle


\section*{Introduction}
Digital cameras are unable to capture the full dynamic range of real scenes (especially those with sunlight). We can capture the full dynamic range of these real scenes by taking multiple exposures.\\

\noindent
HDR images have a large dynamic range. That is, pixels aren't limited to one of 256 values. In some cases, each pixel is represented by a single precision floating-point number.\\

\noindent
In this assignment we'll look at combining images taken at different exposure lengths to form a \textit{high dynamic range} (HDR) image.  In addition, we'll look at how to convert an HDR image to a standard dynamic range (SDR) image.\\


\section*{Grading}
\begin{table}[h]
\begin{centering}
\begin{tabular}{|l|l|}
\hline
Theory Questions & 15pts \\
Plotting pixel value vs log exposure & 15pts\\
Finding Response Curves & 30pts\\
Generating HDR Images & 20pts\\
Tonemapping HDR Images & 20pts\\
\hline
\textbf{TOTAL} & 100pts\\
\hline
\end{tabular}
\caption{Grading Rubric}
\end{centering}
\end{table}

\newpage
\section{(15pts) Theory Questions}
\begin{enumerate}
\item (5pts) Apply a $3\times3$ mean filter to the following 2D matrix.  You may assume that the filter is only applied to areas of the data that have a full 9 samples to process.  Feel free to use Matlab to help you compute this, however, realize that you may be asked to do this without a calculator on an exam.
$$
\begin{bmatrix}
7&   7&     6&     3&     3&     4&     2&     2\\
3&   7&     2&     6&     4&     4&     5&     7\\
5&   4&     7&     5&     1&     1&     2&     2\\
2&   1&     3&     4&     1&     3&     5&     6\\
6&   2&     2&     7&     4&     2&     5&     4\\
2&   2&     2&     3&     6&     6&     6&     7\\
4&   6&     5&     6&     7&     3&     4&     1\\
5&   2&     4&     6&     1&     4&     1&     4\\
\end{bmatrix}
$$

\item (5pts) What is the kernel function for a $5\times5$ Gaussian function with $\sigma=1$?   Normalize the filter so that its elements sum to one.  Feel free to use Matlab to help you compute this, however, realize that you may be asked to do this without a calculator on an exam (leaving things in terms of $e$).


\item (5pts) Given the following 2D kernels, what is the magnitude and direction of the gradient at the center pixel in $I$?  Feel free to use Matlab to help you compute this, however, realize that you may be asked to do this without a calculator on an exam.
\begin{equation}
\frac{\partial}{\partial x} = \begin{bmatrix}
-1/3 & 0 & 1/3\\
-1/3 & 0 & 1/3\\
-1/3 & 0 & 1/3\\
\end{bmatrix}
\end{equation}
\begin{equation}
\frac{\partial}{\partial y} = \begin{bmatrix}
-1/3 & -1/3 & -1/3\\
0 & 0 & 0\\
1/3 & 1/3 & 1/3\\
\end{bmatrix}
\end{equation}
\begin{equation}
I=
\begin{bmatrix}
7 & 7 & 6\\
3 & 7 & 2\\
5 & 4 & 7\\
\end{bmatrix}
\end{equation}

\end{enumerate}


\newpage
\section{(15 points) Plotting pixel value vs log exposure}
On BBlearn you have been provided with a directory, \emph{memorial}.  This directory contains a file \emph{images.txt} that provides a list of images in that directory, as well as their exposure lengths.  Your first task will be to parse the image.txt file to get the list of file names and associated exposure times, and then load all the images in that directory.\\

\noindent
Next, select $three$ pixel locations and plot the values in their \emph{red channel} as a function of $\Delta t_j$, where $\Delta t_j$ is the exposure length for image $j$.  This is akin to plotting the \emph{log irradiance} as a function of the exposure length, but with an identity log irradiance function.\\

\noindent
You image should look something like Figure \ref{fig1}.

\begin{figure}[H]
\begin{center}
\includegraphics{z_vs_t.png}
\caption{Observed red intensity vs Exposure Length}
\label{fig1}
\end{center}
\end{figure}

\newpage

\section{(30 points) Finding and plotting the Log Irradance Functions}
Using the technique discussed in class, find the log irradiance function $g(z_{ij})$ for each color channel.  Then repeat the plot from the previous section.  You image should look something like Figure \ref{fig2}.\\

\noindent
\emph{NOTE:  The more pixels you use to solve the system the better.  That being said, the more pixels you use the larger the matrix to invert will becomes.   Therefore experiment with how many pixels to use.}

\begin{figure}[H]
\begin{center}
\includegraphics{logirradiance_learned.png}
\caption{Log Irradiance vs Exposure Length with learned log irradiance function}
\label{fig2}
\end{center}
\end{figure}

\newpage

\section{(20 points) Generate HDR Images}
Now that we have our log irradiance functions we can combine our images taken with different exposure times!\\

\noindent
For each color channel, go through all the pixel locations and compute the new pixel value to be the average of the pixel's irradiance values from the different exposure length images (making use of the associated channel's log irradiance function, and that image's exposure time).\\  

\noindent
As a quick reference, from the lecture slides, the equations to do this are:

\begin{equation}
ln(R_i) = \frac{1}{P} \sum_{j=1}^P(g(z_{ij})-ln(\Delta t_j))
\end{equation}

\begin{equation}
R_i = e^{ln(R_i)}
\end{equation}

\newpage

\section{(20 points) Tone Mapping an HDR Image}
HDR images cannot be viewed on devices that only support low dynamic range. To be able to view all the details of dark and bright areas at once, the image must be \emph{tonemapped}. \\

\noindent
Tonemap each channel of your HDR image by compressing its values using the compresion function $f(x)=\frac{x}{1+x}$ , then scaling its values to $[0,255]$, then casting it as an unsigned 8-bit integer. 

\newpage
\section*{Submission}
For your submission, upload to Blackboard a single zip file containing:

\begin{enumerate}
\item PDF writeup that includes:
\begin{enumerate}
\item Your answer to the theory question(s).
\item Your plot for Part 2.
\item Your plot for Part 3.
\item Your HDR image for Part 4.
\item Your HDR$\rightarrow$SDR tonemapped image for Part 5.
\end{enumerate}
\item A README text file (\textbf{not} Word or PDF) that explains:
\begin{enumerate}
\item Features of your program
\item Name of your entry-point script
\item Any instructions on how to run your script
\end{enumerate}
\item Your source files.
\end{enumerate}
\end{document}

